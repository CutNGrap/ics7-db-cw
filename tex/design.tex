% !TeX spellcheck = ru_RU
\chapter{Конструкторский раздел}

В данном разделе на основе диаграммы сущность-связь в нотации Чена проводится формализация сущностей проектируемой базы данных, описывается ролевая модель. Также в разделе создается схема триггера, используемого в системе.

%\section{Формализация сущностей системы}
%На рисунке \ref{img:ER} представлена ER-диаграмма сущностей в нотации Чена.
%
%\img{0.5}{ER}{ER-диаграмма в нотации Чена}
%
%На основе ER-диаграммы  спроектированы таблицы базы данных.

\subsection*{Таблица Group}

Содержит информацию об академических группах и включает следующие поля

\begin{enumerate}
	\item id --- идентификатор группы, являющийся первичным ключом;
	\item group\_num --- номер группы (уникальный для потока), целочисленный тип;
	\item faculty --- название факультета, символьный тип;
	\item qualification --- квалификация (бакалавр, специалист и т.д.), символьный тип;
	\item creation --- год создания группы, целочисленный тип.
\end{enumerate}


\subsection*{Таблица Student}

Содержит информацию о студентах и включает следующие поля:

\begin{enumerate}
	\item id --- идентификатор студента, являющийся первичным ключом;
	\item group --- группа студента, внешний ключ;
	\item FIO --- ФИО студента, символьный тип;
	\item book\_num --- номер зачетной книжки студента (уникальный для года поступления), целочисленный тип;
	\item birth --- дата рождения, тип-дата;
	\item enrollment --- год поступления, целочисленный тип.
\end{enumerate}

\subsection*{Таблица Theme}

Содержит информацию о темах курсовых проектов и включает следующие поля:

\begin{enumerate}
	\item id --- идентификатор темы, являющийся первичным ключом;
	\item name --- название темы, символьный тип;
	\item complexity --- сложность темы по 10-бальной шкале, целочисленный тип;
	\item first\_time --- год первой реализации темы, целочисленный тип;
\end{enumerate}

\subsection*{Таблица Source}

Содержит информацию о литературных источников курсовых проектов и включает следующие поля:

\begin{enumerate}
	\item id --- идентификатор источника, являющийся первичным ключом;
	\item name --- название источника, символьный тип;
	\item type --- тип источника (учебник, статья и т.д.), символьный тип;
	\item author --- список авторов, символьный тип;
	\item creation --- год создания источника, целочисленный тип;
\end{enumerate}

\subsection*{Таблица Project}

Содержит информацию о курсовых проектах и включает следующие поля:

\begin{enumerate}
	\item id --- идентификатор проекта, являющийся первичным ключом;
	\item theme\_id --- тема проекта, внешний ключ;
	\item author\_id --- автор проекта, внешний ключ;
	\item mark --- оценка проекта, целочисленный тип;
	\item passed --- дата сдачи проекта, тип-дата;
\end{enumerate}


\section{Ролевая модель}

Ролевая модель в разрабатываемом программном обеспечении необходима для того, чтобы обеспечить правильную организацию работы пользователей в системе. Ее задача заключается в предоставлении каждому пользователю необходимое ему множество операций в системе.

В разрабатываемом программной продукте выделены следующие роли:
\begin{enumerate}
\item StUser --- студент. Имеет доступ SELECT над таблицей Theme и доступ SELECT над таблицей Group.

\item TeUser --- преподаватель. Имеет доступ SELECT к таблицам Theme, Group, Student, Source, Project.



\item AdmUser --- администратор. Имеет все права над Theme, Group, Student, Source, Project.
\end{enumerate}

\section{Разработка триггера}
%В разрабатываемой системе предусмотрена статистическая функция, возвращающая информацию об авторах курсовых проектов, количество использованных источников в которых выше среднего значения количества источников на проект.

В разрабатываемой системе представлен INSERT/UPDATE триггер, который при добавлении в базу нового курсового проекта или изменения уже существующего, проверяет, что год его сдачи больше или равен году первой реализации соответствующей темы. В противном случае триггер меняет год первой реализации данной темы на год сдачи проекта. 

Данный триггер разработан с целью поддержания целостности данных, избавления от возможных противоречий в таблицах.
\clearpage
Схема триггера представлена на рисунке \ref{img:trigger}
\newline

\img{0.6}{trigger}{Схема триггера}


\section{Вывод}
В данном разделе на основе диаграммы сущность-связь в нотации Чена произведена формализация сущностей проектируемой базы данных, описаны все атрибуты, необходимые для представления сущностей предметной области. Также дано полное описание выделенных в системе ролей пользователей и их прав доступа. Была создана схема разрабатываемого INSERT/UPDATE триггера базы данных, выполняющего проверку целостности данных.

