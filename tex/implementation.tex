% !TeX spellcheck = ru_RU
\chapter{Технологический раздел}
В данном разделе рассмотрены средства реализации программного продукта, приведены листинги кода реализованных функций и сценариев, а также пользовательский интерфейс

\section{Выбор СУБД} 

В качестве наиболее популярных реляционных СУБД чаще всего выделяют SQLite \cite{sqlite}, MySQL \cite{mysql} и PostgreSQL \cite{postgres}. Рассмотрим их сильные и слабые стороны.

\begin{enumerate}
	\item SQLite. Данная СУБД является файловой, то есть все данные хранятся в одном файле, что облегчает перемещение и использование в небольших приложениях. Но в связи с этим данная СУБД не оптимизирована для работы с многопользовательскими приложениями и большими объемами данных.
	\item MySQL. Среди преимуществ данной СУБД часто выделяют ее простоту, масштабируемость и сравнительно высокую скорость работы. К минусам данной СУБД относят неполную SQL-совместимость, так как MySQL реализует не весь функционал SQL, и сравнительно невысокую оптимизацию процессов параллельного чтения.
	\item PostgreSQL. В качестве главных преимуществ данной СУБД приводят полную SQL-совместимость, расширяемость и поддержку многими сторонними инструментами, связанными с СУБД. Среди недостатков выделяют операции чтения, в которых PostgreSQL может проигрывать конкурентам.
\end{enumerate}

Для реализации программного продукта была выбрана СУБД PostgreSQL, так как она обладает всем необходимым для проекта функционалом.

\section{Средства реализации}

Для реализации программного продукта был выбран язык программирования Python \cite{python}. Данный язык программирования предоставляет весь необходимый для реализации проекта функционал.

\section{Создание сущностей базы данных}

Создание сущностей в соответствии с результатами проектирования базы данных представлены на листинге \ref{lst:models}.

\lst{10}{54}{utils/models.py}{models}{Создание сущностей базы данных}{Python}

\section{Создание триггера}

Создание INSERT/UPDATE триггера над таблицей Project в соответствии с результатами проектирования представлено в листинге \ref{lst:trigger}.

\lst{23}{26}{scripts/trigger.sql}{trigger}{Создание INSERT/UPDATE триггера над таблицей Project}{SQL}

Для работы триггера также была написана функция \textit{check\_first\_date}, выполняющей проверку года первой сдачи темы курсовой работы в соответствии с результатами проектирования. Ее код приведен в листинге \ref{lst:func}

\lst{1}{18}{scripts/trigger.sql}{func}{Код функции проверки }{SQL}

\section{Создание ролевой модели}

В соответствии с разработанной в конструкторском разделе ролевой моделью был написан SQL-сценарий, выполняющий создание ролей в базе данных и выделение им прав. Код сценария представлен в листинге \ref{lst:role}.

\lst{1}{18}{scripts/roles.sql}{role}{Код сценария создания ролей базы данных и выделения им прав}{SQL}

\section{Интерфейс ПО}

Для работы с базой данных был реализован интерфейс взаимодействия API \cite{API}. Для создания интерфейса была использована библиотека fastapi \cite{fastapi}.

Для каждой сущности базы данных были реализованы операции чтения, добавления, обновления и удаления.

Интерфейс приведен на рисунке \ref{img:api}.

\img{0.75}{api}{Интерфейс взаимодействия с базой данных}

\section*{Вывод}
В данном разделе были рассмотрены средства реализации программного продукта, приведены листинги кода реализованных функций и сценариев, а также пользовательский интерфейс