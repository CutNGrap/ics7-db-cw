% !TeX spellcheck = ru_RU
\chapter{Технологический раздел}
В данном разделе рассматриваются средства реализации программного продукта. Проводится анализ наиболее популярных реляционных СУБД, среди них выбирается наиболее подходящая для разработки программного продукта система. Также в разделе приводится обоснование выбора языка программирования и соответствующей библиотеки создания пользовательского интерфейса.

На основе спроектированной в предыдущих разделах ролевой модели создается ее реализация. Предоставляется код для создания ролей и выдачи им прав, а также создания триггера. Приводится пример работы разработанного приложения.

\section{Выбор СУБД} 

В качестве наиболее популярных \cite{most_popular} реляционных СУБД чаще всего выделяют SQLite \cite{sqlite}, MySQL \cite{mysql} и PostgreSQL \cite{postgres}. Рассмотрим их сильные и слабые стороны.
%
%\begin{enumerate}
%	\item SQLite. Данная СУБД является файловой, то есть все данные хранятся в одном файле, что облегчает перемещение и использование в небольших приложениях. Но в связи с этим данная СУБД не оптимизирована для работы с многопользовательскими приложениями и большими объемами данных.
%	\item MySQL. Среди преимуществ данной СУБД часто выделяют ее простоту, масштабируемость и сравнительно высокую скорость работы. К минусам данной СУБД относят неполную SQL-совместимость, так как MySQL реализует не весь функционал SQL, и сравнительно невысокую оптимизацию процессов параллельного чтения.
%	\item PostgreSQL. В качестве главных преимуществ данной СУБД приводят полную SQL-совместимость, расширяемость и поддержку многими сторонними инструментами, связанными с СУБД. Среди недостатков выделяют операции чтения, в которых PostgreSQL может проигрывать конкурентам.
%\end{enumerate}

Выделим следующие критерии для сравнения выбранных СУБД:
\begin{enumerate}
	\item возможность создания пользователей на уровне БД;
	\item активная поддержка;
	\item полная SQL-совместимость.
\end{enumerate}

Результаты сравнения выбранных СУБД по заданным критериям представлены в таблице \ref{tbl:compare_DBMS}.


\begin{table}[ht!]
	\centering
	\caption{Сравнение выбранных СУБД}
	\label{tbl:compare_DBMS}
	\begin{tabular}{|l|l|l|l|}
		\hline
		\textbf{Критерий} & \textbf{SQLite}& \textbf{MySQL} & \textbf{PostgreSQL}  \\ \hline
		
		1 & - & + & + \\ \hline
		2 & + & - & + \\ \hline
		3 & - & - & +  \\ \hline
		
	\end{tabular}
\end{table}

По результатам сравнения для реализации программного продукта была выбрана СУБД PostgreSQL.

\section{Средства реализации}

Для реализации программного продукта был выбран язык программирования Python \cite{python}. Простота разработки и малый объем кода, обилие библиотек и гибкость данного языка позволяют реализовать графический интерфейс и необходимый для работы с БД объем функций. 

Для связи Python с СУБД PostgreSQL выбрана библиотека SQLAlchemy \cite{sqlalch}. Данная библиотека, как и любая другая библиотека python может быть установлена при помощи встроенного пакетного менеджера pip.


%\section{Создание сущностей базы данных}
%
%Создание сущностей в соответствии с результатами проектирования базы данных представлены на листинге \ref{lst:models}.
%
%\lst{10}{54}{utils/models.py}{models}{Создание сущностей базы данных}{Python}

%\section{Создание функции}
%Создание функции, возвращающей информацию об авторах курсовых проектов, количество использованных источников в которых выше среднего значения количества источников на проект, представлено в листинге \ref{lst:func}.
%
%\lst{29}{44}{scripts/trigger.sql}{trigger}{Создание функции, возвращающей информацию об авторах курсовых проектов, количество использованных источников в которых выше среднего значения количества источников на проект}{SQL}

\section{Создание триггера}

Создание INSERT/UPDATE триггера над таблицей Project в соответствии с результатами проектирования представлено в листинге \ref{lst:trigger}.
\lst{23}{26}{scripts/trigger.sql}{trigger}{Код функции проверки года первой реализации темы}{SQL}

%\begin{center}
%	\begin{lstlisting}[label=lst:trigger, caption=Создание INSERT/UPDATE триггера над таблицей Project]
%create TRIGGER check_first_date_trigger
%AFTER INSERT or UPDATE  ON "project"
%FOR EACH ROW
%EXECUTE FUNCTION check_first_date();
%	\end{lstlisting}
%\end{center}



Для работы триггера также была написана функция \textit{check\_first\_date}, выполняющей проверку года первой сдачи темы курсовой работы в соответствии с результатами проектирования. Ее код приведен в листингах \ref{lst:func}~--~\ref{lst:func1}.


\lst{1}{13}{scripts/trigger.sql}{func}{Код функции проверки года первой реализации темы}{SQL}
\lst{14}{18}{scripts/trigger.sql}{func1}{Код функции проверки года первой реализации темы (продолжение)}{SQL}

%\begin{center}
%	\begin{lstlisting}[label=lst:func, caption=Код функции проверки года первой реализации темы]
%CREATE FUNCTION check_first_date() RETURNS TRIGGER AS $$
%declare first_t int;
%begin
%	select  first_time
%	from theme
%	where id = new.theme_id
%	into first_t;
%	
%	IF first_t > extract('Year' from new.passed) then
%		RAISE NOTICE 'Previous first_time: %', first_t;
%		RAISE NOTICE 'New first_time: %', extract('year' from new.passed);
%		RAISE NOTICE 'Updating table theme with new value: %', extract('year' from new.passed);
%		update theme set first_time = extract('year' from new.passed) where id = new.theme_id;
%		Return NEW;
%	ELSE
%		RETURN NULL;
%END IF;
%END;
%$$ LANGUAGE plpgsql;
%	\end{lstlisting}
%\end{center}

\section{Создание ролевой модели}

В соответствии с разработанной в конструкторском разделе ролевой моделью был написан SQL-сценарий, выполняющий создание ролей в базе данных и выделение им прав. Код сценария представлен в листинге \ref{lst:role}.

\lst{1}{18}{scripts/roles.sql}{role}{Код сценария создания ролей базы данных и выделения им прав}{SQL}

%\begin{center}
%	\begin{lstlisting}[label=lst:role, caption=Код сценария создания ролей базы данных и выделения им прав]
%CREATE ROLE StUser LOGIN PASSWORD 'postgres';
%GRANT SELECT ON TABLE "theme", "grouptab"  TO StUser;
%
%CREATE ROLE TeUser LOGIN PASSWORD 'postgres';
%GRANT SELECT ON TABLE "theme", "grouptab", "student", "source", "project", "source_project"  TO TeUser;
%
%CREATE ROLE AdmUser LOGIN PASSWORD 'postgres';
%GRANT ALL PRIVILEGES ON ALL TABLES IN SCHEMA public TO AdmUser;
%	\end{lstlisting}
%\end{center}

\section{Пользовательский интерфейс}

Для работы с базой данных был реализован графический интерфейс пользователя. Для создания интерфейса была использована библиотека pyqt5 \cite{pyqt5}.

Для каждой сущности базы данных были реализованы операции чтения, добавления, обновления и удаления.

Для операций чтения и удаления необходимо указать идентификатор записи в соответствующей таблицы. Для операций обновления и добавления необходимо указать значения полей, соответствующих атрибутам целевой таблицы.

\pagebreak 

Для операций массового чтения также необходимо указать количество элементов на возвращаемой странице и количество страниц, которые необходимо пропустить при выводе.

Пример работы программы приведен на рисунке \ref{img:example}.

\img{0.5}{example}{Пример работы программы}

\section*{Вывод}
В данном разделе рассмотрены наиболее популярные СУБД, среди которых выбрана система,наилучшим образом подходящая для решения поставленной задачи. Приведено обоснование выбор средств реализаций, представлен код создания триггера и ролевой модели. Описан пользовательский интерфейс и приведен пример работы приложения.