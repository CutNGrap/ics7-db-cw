% !TeX spellcheck = ru_RU
\chapter{Исследовательский раздел}

В данном разделе поставлена задача исследования зависимости времени выполнения запросов, выполняющих подсчет количества записей в таблице базы данных, от наличия в базе данных индекса для соответствующего атрибуты таблицы.

Также в разделе приведены технические характеристики устройства, на котором проводилось тестирование, приведены результаты замеров времени в табличном и графическом виде. Сделан вывод о рациональности использования индекса в зависимости от синтаксиса запроса.


\section{Технические характеристики}

Замеры времени одних и тех же операций могут сильно различаться в зависимости от технических характеристик устройства, на котором проводится тестирование.
Поэтому для качественного проведения исследования временных свойств системы необходимо знать основные параметры тестирующего устройства. 

Ниже приведены технические характеристики устройства, на котором было проведено измерение времени работы ПО:

\begin{itemize}
	\item операционная система Windows 10 Домашняя Версия 21H1 x86\_64;
	\item оперативная память 8 Гбайт 2133 МГц;
	\item процессор Intel Core i5-8300H с тактовой частотой 2.30 ГГц, 4 физических ядра, 8 логических ядер.
\end{itemize}

\section{Постановка задачи исследования}
Индекс \cite{index} в базе данных --- любая структура данных, позволяющая сократить время нахождения записи в БД по одному или нескольким значениям полей отношения. Сокращение времени достигается за счет дополнительных затрат памяти на хранение структур данных. Также повышается время  вставки, изменения и удаления, так как при выполнении данных операций системе необходимо вносить изменения в индексы для поддержания их в актуальном состоянии.

Целью исследования является изучение зависимости времени выполнения запросов подсчета количества записей в таблице базы данных от наличия соответствующего индекса базы данных.

Для исследования были выбраны следующие запросы:
\begin{enumerate}
	\item Запрос, выполняющий подсчет общего числа студентов, обучающихся в группе со значением id = 50. Код данного запроса представлен в листинге \ref{lst:sql1}.
	\item Запрос, выполняющий подсчет общего числа студентов в соответствующей таблице. Код данного запроса представлен в листинге \ref{lst:sql2}.
\end{enumerate}

\begin{lstlisting}[label=lst:sql1, caption=Листинг запроса\, выполняющего поиск числа студентов\, обучающихся в группе со значением  id равном 50]
select count(group_id) from student where group_id = 50
\end{lstlisting}

\begin{lstlisting}[label=lst:sql2, caption=Листинг запроса\, выполняющего поиск общего числа студентов в соответствующей таблице]
select count(group_id) from student 
\end{lstlisting}

Также для исследования был использован индекс для атрибута \textit{group\_id} таблицы студентов.

Исследование проводилось для состояний таблицы студентов, содержащих $10^5, 2\times10^5, 3\times10^5,  4\times10^5, 5\times10^5, 6\times10^5, 7\times10^5, 8\times10^5, 9\times10^5, 10^6$ студентов.

Для каждого размера таблицы было проведено 2 измерения:  время выполнения запросов без использования индекса и с его использованием.

\clearpage
\section{Время выполнения запросов}

В таблицах \ref{tab:time} -- \ref{tab:time2} продемонстрировано пользовательское время выполнения запросов с использованием индекса и без него.

\begin{table}[ht!]
	\begin{center}
		
		\caption{Время выполнения запроса 1}
		\label{tab:time}
		\begin{tabular}{|c|c|c|}
			\hline
			Количество студентов & Время без индекса, мс & Время с индексом, мс \\
			\hline
			$10^5$    & 12.47   & 7.41  \\
			\hline
			$2\times10^5$   & 19.62   &  8.31 \\
			\hline
			$3\times10^5$     & 26.78   & 9.18 \\
			\hline
			$4\times10^5$     & 31.43  &   10.08 \\
			\hline
			$5\times10^5$    & 33.60    & 10.84 \\
			\hline
			$6\times10^5$       & 36.05   & 11.75  \\
			\hline
			$7\times10^5$      & 38.85   &  12.63 \\
			\hline
			$8\times10^5$     & 44.15  &  13.24 \\
			\hline
			$9\times10^5$    & 47.17  & 14.08 \\
			\hline
			$10^6$     & 49.50 & 15.53 \\
			\hline
		\end{tabular}
	\end{center}
\end{table}

\begin{table}[ht!]
	\begin{center}
		
		\caption{Время выполнения запроса 2}
		\label{tab:time2}
		\begin{tabular}{|c|c|c|}
			\hline
			Количество студентов & Время без индекса, мс & Время с индексом, мс \\
			\hline
			$10^5$    & 12.44   & 13.27  \\
			\hline
			$2\times10^5$   & 18.68   &  20.83 \\
			\hline
			$3\times10^5$     & 26.67   & 27.63 \\
			\hline
			$4\times10^5$     & 30.88  &   30.23 \\
			\hline
			$5\times10^5$    & 33.45    & 34.84 \\
			\hline
			$6\times10^5$       & 35.55   & 36.14  \\
			\hline
			$7\times10^5$      & 38.44   &  39.77\\
			\hline
			$8\times10^5$     & 42.70  &  43.35 \\
			\hline
			$9\times10^5$    & 46.89  & 46.91 \\
			\hline
			$10^6$     & 49.21 & 49.15 \\
			\hline
		\end{tabular}
	\end{center}
\end{table}

\clearpage

На рисунках \ref{img:Figure_1} -- \ref{img:Figure_2} представлена графическая интерпретация полученных результатов.

\img{0.55}{Figure_1}{Графическая интерпретация результатов исследования \newline времени выполнения запроса 1}
\img{0.55}{Figure_2}{Графическая интерпретация результатов исследования \newline времени выполнения запроса 2}
\pagebreak

Из результатов исследования можно сделать вывод, что время выполнения запроса 1, обладающего явным указанием значения искомого поля, может быть значительно снижено с использованием индекса. Для таблицы из $10^6$ записей время было сокращено более чем в 3 раза. 

В случае запроса 2 различия значений времени выполнения с использованием индекса и без него отличаются слабо и, вероятно, объясняются случайной погрешностью. Также наличие индекса увеличивает объем занимаемого дискового пространства и его создание для выполнения запросов, подобных 2 не рекомендуется.





\section*{Вывод}

В данном разделе было проведено исследование влияния индекса базы данных на производительность запросов вычисления количества записей в таблице. 

Из результатов исследования можно сделать вывод, что в зависимости от структуры запроса использование индекса в базе данных может как уменьшать время выполнения запроса, так и не оказывать на него влияния. В обоих случаях наличие индекса повышает объем занимаемой памяти и увеличивает время работа операций вставки, изменения и удаления.

Рекомендуется применять индексы с осторожностью и проводить анализ оправданности их создания.


