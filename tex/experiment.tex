% !TeX spellcheck = ru_RU
\chapter{Исследовательский раздел}

В данном разделе будут приведены примеры работы разработанной программы и исследована зависимость времени выполнения реализации алгоритма бросания лучей от количества потоков, участвующих в обработке.

\section{Демонстрация работы программы}

На рисунке \ref{img:example} продемонстрированы интерфейс и пример работы программы.

\img{0.4}{example}{Пример работы программы}

Интерфейс программы предоставляет пользователю возможность создавать геометрические примитивы, задавая их геометрические параметры, перемещать, поворачивать и удалять выбранные объекты.

Перемещение камеры реализовано через кнопки в нижней части экрана.

Список объектов сцены и их параметров представлен в правом верхнем углу экрана. Выбор объекта реализован через добавление его номера в поле <<Номер объекта для преобразования>>.

При помощи полей секции <<Параметры текстуры>> пользователь может задавать размер (в пикселях) стороны генерируемой для выбранного объекта текстуры (текстура квадратная), размер камня текстуры, среднее значение цвета камня в палитре HSV.

\section{Технические характеристики}

Ниже приведены технические характеристики устройства, на котором было проведено измерение времени работы ПО:

\begin{itemize}
	\item операционная система Windows 10 Домашняя Версия 21H1 \cite{windows} x86\_64;
	\item оперативная память 8 Гбайт 2133 МГц;
	\item процессор Intel Core i5-8300H с тактовой частотой 2.30 ГГц \cite{intel}, 4 физических ядра, 8 логических ядер.
\end{itemize}

\section{Время выполнения реализации алгоритма}

Для замеров времени использовалась функция получения значения системных часов $clock\_gettime()$ \cite{gettime}. Функция применялась два раза --- в начале и в конце измерения времени, значения полученных временных меток вычитались друг из друга для получения времени выполнения программы.


Замеры проводились по 5 раз для набора значений количества потоков {0, 1, 2, 4, 8, 16, 32, 64}, где значение \textit{0} соответствует однопоточной программе, а значение \textit{1} --- программе, создающей один дополнительный поток, выполняющий все вычисления.

Для каждого потока измерения проводились для размера экрана от $300\times300$ пикселей (90000 лучей) до $2100\times2100$ пикселей (4410 тысяч лучей).

В таблице \ref{tbl:threads} представлены замеры времени выполнения программы в зависимости от количества потоков.


\begin{table}[h]
	\begin{center}
		\begin{threeparttable}
			\caption{Результаты нагрузочного тестирования (в мс)}
			\label{tbl:threads}
			\begin{tabular}{|c|c|c|c|c|c|c|c|c|}
				\hline
				{Кол-во лучей, тыс} & \multicolumn{8}{c}{Кол-во потоков, ед} \\
				\hline
				& 0 & 1 & 2 & 4 & 8 & 16 & 32 & 64 \\
				\hline
				90 & 156 & 160 & 83 & 62 & 39 & 47 & 47 & 48 \\
				\hline
				360 & 610 & 616 & 328 & 214 & 162 & 171 & 171 & 173 \\
				\hline
				810 & 1401 & 1409 & 768 & 494 & 389 & 409 & 411 & 411 \\
				\hline
				1440 & 2469 & 2499 & 1371 & 889 & 689 & 702 & 700 & 701 \\
				\hline
				2250 & 3763 & 3856 & 2040 & 1332 & 1048 & 1057 & 1057 & 1059 \\
				\hline
				3240 & 5655 & 5666 & 3090 & 1861 & 1564 & 1584 & 1606 & 1613 \\
				\hline
				4410 & 7446 & 7516 & 4033 & 2560 & 2114 & 2114 & 2114 & 2115 \\
				\hline
				
			\end{tabular}
		\end{threeparttable}
	\end{center}
\end{table}

%\clearpage
На рисунках \ref{img:all_threads} -- \ref{img:threads_1800} приведена графическая интерпретация результатов замеров.

\img{0.7}{all_threads}{Результаты замеров времени работы реализаций алгоритма с разным количеством потоков в зависимости от количества лучей}
\img{0.7}{threads_1800}{Результаты замеров времени работы реализации алгоритма для 3240 тысяч лучей в зависимости от количества потоков}

\clearpage


Из полученных результатов можно сделать вывод, что однопоточный процесс работает быстрее процесса, создающего один отдельный поток для обработки всех документов. Это связано с дополнительными временными затратами на создание потока и передачи ему необходимых аргументов.

Наилучший результат по времени для всех значений количества документов показал процесс с 8 дополнительными потоками, выполняющими вычисления. Рекомендуется использовать на данной архитектуре ЭВМ число дополнительных потоков равное числу логических ядер устройства.

Для числа потоков, большего 8, затраты на содержание потоков превышают преимущество от использования многопоточности, и функция времени от количества потоков начинает расти.


\section*{Вывод}

В результате исследования было выявлено, что использование многопоточности может уменьшить время выполнения реализации алгоритма по сравнению с однопоточной программой.

Выборка из результатов замеров времени (для 3240 тысяч лучей):
\begin{itemize}
	\item однопоточный процесс --- 5655 мс;
	\item один дополнительный поток, выполняющий все вычисления (худший результат) --- 5666 мс, что в 1,002 раз медленнее времени выполнения однопоточного процесса;
	\item 8 потоков (лучший результат) --- 1564 мс, что в 3,62 раз быстрее времени выполнения однопоточного процесса;
\end{itemize}

Таким образом, использование дополнительных потоков может как ускорить выполнение процесса по сравнению с однопоточным процессом (в 3,62 раз для 8 потоков), так и незначительно увеличить время выполнения (в 1,002 раз для одного дополнительного потока). 

Незначительность увеличения времени, необходимого на содержание дополнительного потока связано с высокой трудоемкостью алгоритма бросания лучей. По сравнению со временем выполнения данного алгоритма временные затраты на выделение потока малы.

Рекомендуется использовать на данной архитектуре ЭВМ число дополнительных потоков равное числу логических ядер устройства.


