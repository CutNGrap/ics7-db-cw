% !TeX spellcheck = ru_RU
\chapter{Аналитический раздел}

В данном разделе представлен анализ существующих систем управления базами данных, проведена формализация информации, подлежащей хранению.

\section{Определение базы данных}
\textbf{База данных}\cite{db} --- это упорядоченный набор структурированных данных, хранящихся в электронном виде. 

По способу применения базы данных разделяют на два типа.
\begin{enumerate}
	\item \textbf{OLAP}(online analytical processing) --- система, используемая для обработки больших объемов данных.
	\item \textbf{OLTP}(online transactional processing) --- система, основной упор которой делается на быструю обработку запросов, операции в режиме реального времени.
\end{enumerate}

В связи с необходимостью работать с базой данных в режиме реального времени в разрабатываемом программной обеспечении будет использована технология OLTP.

\section{Определение СУБД}
\textbf{Система управления базами данных (СУБД)} --- это приложение, обеспечивающее создание, хранение, обновление и поиск информации в базе данных.

Существует множество классификация СУБД, использующие разные признаки классификации. Далее приведены некоторые из них.
\section{Классификация СУБД по модели данных}
\subsection*{Дореляционные СУБД}
\begin{enumerate}
	\item \textbf{Инвертированные списки}. 
	Базы данных, основанные на использовании инвертированных списков представляют собой совокупность файлов, содержащих записи.
	Инвертированный список –-- это организованный в вид списка специального вида индекс файла, позволяющий получить всю совокупность указателей на записи, которым соответствует заданное значение ключа индексации.
	В общем случае ограничения целостности базы данных отсутствуют. В некоторых системах поддерживаются ограничения уникальности значений некоторых полей, но в основном все возлагается на прикладную программу.
	\item \textbf{Иерархическая модель}.
	Иерархические модели имеют древовидную структуру, где каждому узлу соответствует один сегмент, представляющий собой запись (кортеж полей) базы данных. Каждому сегменту может соответствовать несколько дочерних сегментов, однако запись-потомок должна иметь в точности одного предка.
	\item \textbf{Сетевые СУБД}.
	Сетевой подход к организации данных является расширением иерархического. На формирование связи ограничений не накладывается, в отличие от иерархической модели, предполагающей ровно одного предка у записи-потомка.
\end{enumerate}

\subsection*{Реляционные СУБД}

Реляционная модель предполагает организацию данных в виде таблиц (отношений), содержащих информацию о сущностях. Каждая запись таблицы содержит уникальный индекс (ключ), используемый для поиска связанных данных в разных таблицах.

Реляционная модель состоит из трех частей:
\begin{enumerate}
	\item \textbf{структурная часть} фиксирует, что база данных использует только одну структуру данных --- n-арное отношение;
	\item \textbf{целостная часть} описывает ограничения, накладываемые на отношения реляционной модели;
	\item \textbf{манипуляционная часть} описывает два эквивалентных способа манипулирования реляционными данными --- реляционную алгебру и реляционное исчисление.

\end{enumerate}

\subsection*{Постреляционные СУБД}

Постреляционная модель представляет собой обобщене реляционной модели. Она допускает многозначные поля таблиц, каждое из которых рассматривается как самостоятельное отношение, встроенное в главное отношение. 

\section{Классификация СУБД по способу доступа к базе данных}

\begin{enumerate}
	\item \textbf{Файл-серверные СУБД}.
	Значительная часть вычислений выполняется на стороне клиента, сервер отвечает только за извлечение данных из файлов и отправку пользователю.
	\item \textbf{Клиент-серверные СУБД}.
	 Основная вычислительная нагрузка ложится на сервер базы данных, задача клиента заключается в предварительной обработке данных и организации доступа к серверу.
	\item \textbf{Встраиваемые СУБД.} 
	СУБД представляет собой библиотеку, позволяющую структурировать и хранить большие объемы данных на локальной машине. 
	\item \textbf{Сервис-ориентированные СУБД.} 
	База данных представляет собой хранилище сообщений и метаинформации о сервисах и очередях сообщений.
	
	\item \textbf{Другие СУБД.}

\end{enumerate}	
\section{Классификация СУБД по организации хранения данных}
\begin{enumerate}
	\item \textbf{Локальные СУБД.} 
	Все части СУБД располагаются на одной машине
	
	\item \textbf{Распределенные СУБД.}
	Части СУБД распределены на двух или более машинах.
\end{enumerate}

\section{Формализация данных}

База данных состоит из следующих сущностей:
\begin{enumerate}
	\item таблица курсовых проектов Project;
	\item таблица литературных источников Source;
	\item таблица академических групп Group;
	\item таблица студентов Student;
	\item таблица тем курсовых проектов Theme;
\end{enumerate}

На рисунке \ref{img:ER} представлена ER-диаграмма сущностей в нотации Чена.

\img{0.5}{ER}{ER-диаграмма в нотации Чена}

\section{Формализация пользователей}

В системе присутствуют три уровня пользователей.

\begin{enumerate}
	\item \textbf{Студент} --- пользователь, обладающий возможностями просмотра сущностей академических групп и тем курсовых проектов.
	\item \textbf{Преподаватель} --- пользователь, обладающий возможностями просмотра всех сущностей, перечисленных в разделе <<Формализация данных>>.
	\item \textbf{Администратор} --- пользователь, обладающий возможностями изменения сущностей и полей базы данных, просмотра всех сущностей.
\end{enumerate}

\section{Анализ существующих решений}
Среди уже существующих проектов были выделены 3 аналога, частично решающие поставленную задачу. Их сравнительный анализ представлен в таблице \ref{tbl:anal}.

Для сравнения были выбраны следующие критерии:
\begin{enumerate}
	\item группы --- возможность просмотра информации об академических группах студентов;
	\item антиплагиат --- невозможность приобретения готового проекта другого человека;
	\item темы --- возможность просмотра тем уже существующих работ для поиска вдохновения.
\end{enumerate}

\begin{table}[ht!]
	\centering
	\caption{Существующие решения поставленной задачи}
	\label{tbl:anal}
	\begin{tabular}{|c|c|c|c|}
		\hline
		\textbf{Название проекта} & \textbf{Группы} & \textbf{Антиплагиат} & \textbf{Темы}\\
		\hline
		
		\textbf{ЭУ МГТУ}\cite{eu} & + 
		& +
		& -\\
		\hline
		
		\textbf{Studynote} \cite{studynote} & - 
		& +
		& +\\
		\hline
		
		\textbf{Workspay \cite{workspay}} & -
		& -
		& +\\
		\hline
	\end{tabular}
\end{table}

Из таблицы можно сделать вывод, что ни один из перечисленных аналогов не обладает функционалом, удовлетворяющим требованиям или обладает лазейками для недобросовестного выполнения курсовой работы.

Создаваемое программное обеспечение будет предоставлять описанный функционал, являясь некоммерческим продуктом и не предоставляя возможности для плагиата.

\section{Выбор модели данных}

Для реализации программного продукта была выбрана реляционная модель данных, так как она обладает рядом преимуществ в рамках проекта:

\begin{enumerate}
	\item позволяет реализовать связи между выделенными сущностями и исключить дублирование за счет использования механизмов первичного и внешнего ключа;
	\item подразумевает хранение в виде таблиц, понятных конечному пользователю;
	\item имеет возможность произвольного доступа ко всем элементам сущностей.
\end{enumerate}

\section{Вывод}
В данном разделе представлен анализ существующих систем управления базами данных, проведена формализация информации, подлежащей хранению.