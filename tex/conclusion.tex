% !TeX spellcheck = ru_RU
\chapter*{ЗАКЛЮЧЕНИЕ}
\addcontentsline{toc}{chapter}{ЗАКЛЮЧЕНИЕ}
 
В ходе выполнения работы были рассмотрены различные ансамблевые алгоритмы машинного обучения --- инструментарий для создания сильных учеников на основе конечного числа слабых моделей.

В работе приведены возможные применения алгоритмов в контексте обработки фотографий с околоземной орбиты с учетом их сильных сторон. Аапример,  алгоритм бэггинга применим для задач классификации объектов на изображении, флгоритм стекинга позволяет реализовать мультиспектральную обработку изображений для анализа данных вне диапазона человеческого цветовосприятия, а алгоритм стекинга засчет итеративной обработки фотографии подходит для улучшения качества изображения и удаления шумов.

Однако определить однозначно лучший алгоритм не представляется возможным. Рекомендуется выбор используемого ансамблевого алгоритма на основе проведения анализа поставленной задачи с учетом особенностей обрабатываемых данных, программной доступности и требований заказчика.



Цель, поставленная в начале работы, была достигнута. Кроме того были достигнуты все поставленные задачи:
\begin{enumerate}
	\item были описаны ансамблевые алгоритмы машинного обучения;
	\item были выбраны критерии классификации;
	\item было произведено сравнение этих методы в контексте обработки фотографий с околоземной орбиты.
\end{enumerate}

